
%% ProcessMusic.tex
%% V0.1
%% 2012/10/23
%% by Kyle Kastner
%%
%% requires IEEEtran.cls version 1.7 or later
%% Thsi file based on content from http://www.ctan.org/tex-archive/macros/latex/contrib/IEEEtran/

\documentclass[journal]{IEEEtran}
\usepackage[cmex10]{amsmath}
\begin{document}
\title{ProcessMusic: A DSP Application Framework}

\author{Kyle Kastner and Johanna Hansen}%

\markboth{Project 1: DSP Framework}{}
\maketitle

\begin{abstract}
DSP applications require extensive debugging and testing after initial development. However, most environments 
for DSP algorithm development and visualization lack performance for large-scale use.
This framework provides basic DSP functions in performant, compiled code, while also having a dynamic
interface for visualization and high-level computation, in order to bridge the gap between development and deployment. 
\end{abstract}
% IEEEtran.cls defaults to using nonbold math in the Abstract.
% This preserves the distinction between vectors and scalars. However,
% if the journal you are submitting to favors bold math in the abstract,
% then you can use LaTeX's standard command \boldmath at the very start
% of the abstract to achieve this. Many IEEE journals frown on math
% in the abstract anyway.

% Note that keywords are not normally used for peerreview papers.
\begin{IEEEkeywords}
Python, C++, DSP, Numpy.
\end{IEEEkeywords}

\IEEEpeerreviewmaketitle
\section{Introduction}
\IEEEPARstart{T}{he fields} of digital signal processing and software development are highly related. As computing power
grows, the amount of data to process also grows larger. Algorithms become increasingly more elaborate, and the amount of time, knowledge,
and effort necessary to create a useful application has begun to eclipse the cost for hardware itself.

\subsection{Background}
Work in DSP often requires both standard debugging tools and expert verification of the processing stages involved 
in the data pipeline. As a result, algorithm development and deployment tend to occur in two separate stages.
First, the algorithm is developed in a high-level language such as MATLAB or Python, using tools for visualization 
and basic debugging in order to verify the operations work as expected. Next, the same application is ported to a low-level compiled
language, in order to meet the performance required for most DSP applications. During this translation, many new and
unforseen bugs occur, lengthening development time, and increasing project complexity. 

\subsection{Overview}
By combining performant data processing code blocks and a high-level scripting interface into a unified toolkit, it 
is possible to utilize the strengths of both dynamic and compiled code, reducing the difficulty and hassle of DSP application 
development and deployment. This paper documents the driving concerns and specific goal applications that lead the development of this toolkit. 
It also explores design, capabilities, example applications, and future work. 

\subsection{Goal}
Subsection text here.

\subsection{Key Concerns}
Subsection text here.
\subsubsection{Performance}
Subsubsection text here.
\subsubsection{Visualization}
Subsubsection text here.
\subsubsection{Flexibility}
Subsubsection text here.

\section{Design}
Subsection text here.

\subsection{Implementation}
Subsection text here.
\subsubsection{C++}
Subsection text here.
\subsubsection{Python}
Subsection text here.

\section{Capabilities}
Subsection text here.

\subsection{Processing}
\subsubsection{Optional File Decompression}
Subsection text here.
\subsubsection{FFT and Power Density}
Subsection text here.
\subsubsection{Cepstrum}
Subsection text here
\subsubsection{DCT}
Subsection text here.
\subsubsection{Bark Filter Bank}
Subsection text here.
\subsubsection{Gaussian Mixture Modeling}
Subsection text here.

\subsection{Development}
\subsubsection{Automation}
Subsection text here.
\subsubsection{Visualization}
Subsection text here.
\subsubsection{Configuration}
Subsection text here.

\section{Applications}
Subsection text here.

\section{Future Work}
Subsection text here.
\subsection{DSP}
\subsection{Software Capabilities}

\section{Conclusion}
The conclusion goes here.

% if have a single appendix:
%\appendix[Proof of the Zonklar Equations]
% or
%\appendix  % for no appendix heading
% do not use \section anymore after \appendix, only \section*
% is possibly needed

% use appendices with more than one appendix
% then use \section to start each appendix
% you must declare a \section before using any
% \subsection or using \label (\appendices by itself
% starts a section numbered zero.)
%

\appendices
\section{Proof of the First Zonklar Equation}
Appendix one text goes here.

% you can choose not to have a title for an appendix
% if you want by leaving the argument blank
\section{Continued Proof}
Appendix two text goes here.

\begin{thebibliography}{1}

\bibitem{IEEEhowto:kopka}
H.~Kopka and P.~W. Daly, \emph{A Guide to \LaTeX}, 3rd~ed.\hskip 1em plus
  0.5em minus 0.4em\relax Harlow, England: Addison-Wesley, 1999.

\end{thebibliography}

% if you will not have a photo at all:
\begin{IEEEbiographynophoto}{Kyle Kastner}
Kyle Kastner works at Southwest Research Institute, Division 16.
\end{IEEEbiographynophoto}

\begin{IEEEbiographynophoto}{Johanna Hansen}
Johanna Hansen works at Southwest Research Institute, Division 16.
\end{IEEEbiographynophoto}

% You can push biographies down or up by placing
% a \vfill before or after them. The appropriate
% use of \vfill depends on what kind of text is
% on the last page and whether or not the columns
% are being equalized.

%\vfill

% Can be used to pull up biographies so that the bottom of the last one
% is flush with the other column.
%\enlargethispage{-5in}

% that's all folks
\end{document}


